%% This is file `elsarticle-template-1-num.tex',
%%
%% Copyright 2009 Elsevier Ltd
%%
%% This file is part of the 'Elsarticle Bundle'.
%% ---------------------------------------------
%%
%% It may be distributed under the conditions of the LaTeX Project Public
%% License, either version 1.2 of this license or (at your option) any
%% later version.  The latest version of this license is in
%%    http://www.latex-project.org/lppl.txt
%% and version 1.2 or later is part of all distributions of LaTeX
%% version 1999/12/01 or later.
%%
%% The list of all files belonging to the 'Elsarticle Bundle' is
%% given in the file `manifest.txt'.
%%
%% Template article for Elsevier's document class `elsarticle'
%% with numbered style bibliographic references
%%
%% $Id: elsarticle-template-1-num.tex 149 2009-10-08 05:01:15Z rishi $
%% $URL: http://lenova.river-valley.com/svn/elsbst/trunk/elsarticle-template-1-num.tex $
%%
\documentclass[final,12pt, times, 5p, twocolumn]{elsarticle}
\usepackage[brazilian]{babel}
\usepackage[utf8]{inputenc}

\usepackage[T1]{fontenc}

%% Use the option review to obtain double line spacing
%% \documentclass[preprint,review,12pt]{elsarticle}

%% Use the options 1p,twocolumn; 3p; 3p,twocolumn; 5p; or 5p,twocolumn
%% for a journal layout:
%% \documentclass[final,1p,times]{elsarticle}
%% \documentclass[final,1p,times,twocolumn]{elsarticle}
%% \documentclass[final,3p,times]{elsarticle}
%% \documentclass[final,3p,times,twocolumn]{elsarticle}
%% \documentclass[final,5p,times]{elsarticle}
%% \documentclass[final,5p,times,twocolumn]{elsarticle}

%% if you use PostScript figures in your article
%% use the graphics package for simple commands
%% \usepackage{graphics}
%% or use the graphicx package for more complicated commands
%% \usepackage{graphicx}
%% or use the epsfig package if you prefer to use the old commands
%% \usepackage{epsfig}

%% The amssymb package provides various useful mathematical symbols
\usepackage{amssymb}
%% The amsthm package provides extended theorem environments
%% \usepackage{amsthm}

%% The lineno packages adds line numbers. Start line numbering with
%% \begin{linenumbers}, end it with \end{linenumbers}. Or switch it on
%% for the whole article with \linenumbers after \end{frontmatter}.
%% \usepackage{lineno}

%% natbib.sty is loaded by default. However, natbib options can be
%% provided with \biboptions{...} command. Following options are
%% valid:

%%   round  -  round parentheses are used (default)
%%   square -  square brackets are used   [option]
%%   curly  -  curly braces are used      {option}
%%   angle  -  angle brackets are used    <option>
%%   semicolon  -  multiple citations separated by semi-colon
%%   colon  - same as semicolon, an earlier confusion
%%   comma  -  separated by comma
%%   numbers-  selects numerical citations
%%   super  -  numerical citations as superscripts
%%   sort   -  sorts multiple citations according to order in ref. list
%%   sort&compress   -  like sort, but also compresses numerical citations
%%   compress - compresses without sorting
%%
%% \biboptions{comma,round}

% \biboptions{}


\journal{Nuclear Physics B}

\begin{document}

\begin{frontmatter}

%% Title, authors and addresses

%% use the tnoteref command within \title for footnotes;
%% use the tnotetext command for the associated footnote;
%% use the fnref command within \author or \address for footnotes;
%% use the fntext command for the associated footnote;
%% use the corref command within \author for corresponding author footnotes;
%% use the cortext command for the associated footnote;
%% use the ead command for the email address,
%% and the form \ead[url] for the home page:
%%
%% \title{Title\tnoteref{label1}}
%% \tnotetext[label1]{}
%% \author{Name\corref{cor1}\fnref{label2}}
%% \ead{email address}
%% \ead[url]{home page}
%% \fntext[label2]{}
%% \cortext[cor1]{}
%% \address{Address\fnref{label3}}
%% \fntext[label3]{}

\title{Captura de dados do contexto urbano para o uso da bicicleta.}

%% use optional labels to link authors explicitly to addresses:
%% \author[label1,label2]{<author name>}
%% \address[label1]{<address>}
%% \address[label2]{<address>}

\author{Antonio Fernando Santos Ladeia}

\address{Instituto Federal de Educação, Ciência e Tecnologia da Bahia
\\ Rua Emidio dos Santos, s/n, Barbalho -  Bahia.
\\ ladeia@ifba.edu.br}

\begin{abstract}
%% Text of abstract

\end{abstract}

\begin{keyword}
%% keywords here, in the form: keyword \sep keyword

%% MSC codes here, in the form: \MSC code \sep code
%% or \MSC[2008] code \sep code (2000 is the default)

\end{keyword}

\end{frontmatter}

%%
%% Start line numbering here if you want
%%
% \linenumbers

%% main text
\section{Introdução}

\section{Justificativa}
A mobilidade urbana é um ponto de ampla discussão e impacto na sociedade atual. Os espaços reservados aos automóveis multiplicam-se tomando o espaço de transportes coletivos e não motorizados, como caminhar ou pedalar. Com a facilidade de acesso ao financiamento e a isenção de impostos, mais e mais pessoas ocupam as ruas com seus carros,  dificultando o trânsito  e prejudicando a vida da grande maioria.

Na prática, não existem políticas governamentais de que possuam como objetivo legitimar outras formas de mobilidade, como a bicicleta, como um modal de transporte reconhecido, promovendo o seu uso e conscientizando condutores de veículos motorizados sobre as leis de trânsito para um compartilhamento pacífico e amigável das ruas. 

Este contexto gera um ambiente hostil para os usuários de bicicleta e desencorajador para aqueles que gostariam de adotar este modal de locomoção nas cidades. Faltam iniciativas e dados que melhor detalhem e quantifiquem este contexto adverso como:
\begin{itemize}

  \item Altas velocidades nas ruas e avenidas
  \item Falta de sinalização
  \item Ruas arborizadas
  \item ETC. \ldots

\end{itemize}

\section{Objetivos}
Este trabalho possui como objetivo principal a captura de informações reais e detalhadas relativas ao ambiente das ruas a partir do uso de bicicletas, para melhor entender os fatores que fazem as ruas dos grandes centros urbanos tão hostis e desmotivadores deste meio de transporte. 

Busca-se, assim, tratar tanto das questões ambientais como das questões de tráfego urbano e das condições, favoráveis, ou não, para o uso da bicicleta.

Nesta mesma linha de raciocínio, temos como objetivos secundários:

\begin{itemize}

  \item Mapear espacialmente/geograficamente os dados coletados, no sentido de indicar os trechos com condições mais favoráveis ou menos atrativos.
  \item Oferecer subsídios para políticas de adequação do espaço.
  \item A observação da legitimação das políticas públicas de transito por parte de motoristas e ciclistas, avaliando esta relação nas práticas de uso das pistas. \ldots

\end{itemize}

\section{Referencial teórico}

Para entender o contexto onde o trabalho está situado, é importante conhecer alguns conceitos em que ele se fundamenta, por isso, esta sessão tem por objetivo pautar os trabalhos similares e os que serviram de referencial para o mesmo.

\subsection{Software Livre}
O conceito de Software Livre~\cite{softwareLivre} nasce com o projeto GNU~\cite{GNU}. O Software Livre possui enraizada a filosofia de ser um movimento social ao invés de apenas uma decisão técnica. Ela é muito mais que apenas código aberto, ela é uma posição política e filosófica que se opõe ao software proprietário na medida em que o conhecimento gerado com o software proprietário é fechado a poucas pessoas.

Para um \texttt{software} ser considerado livre, ele precisa seguir quatro liberdades, que são os pilares do movimento Software Livre, e aqui, traduzidas livremente:

\begin{itemize}
  \item A liberdade de executar o programa, para qualquer propósito (liberdade nº 0)
  \item A liberdade de estudar como o programa funciona e adaptá-lo para as suas necessidades (liberdade nº 1). O acesso ao código-fonte~\cite{codigoFonte} é um pré-requisito para esta liberdade.
  \item A liberdade de redistribuir cópias de modo que você possa ajudar ao seu próximo (liberdade nº 2).
  \item A liberdade de aperfeiçoar o programa, e liberar os seus aperfeiçoamentos, de modo que toda a comunidade se beneficie deles (liberdade nº 3). O acesso ao código-fonte é um pré-requisito para esta liberdade.
\end{itemize}

\subsection{Hardware livre}
O conceito de Hardware Livre, surgiu apoiado no conceito de Software Livre, mas assim como o Software Livre, não se restringiu apenas a parte técnica dos dispositivos físicos. O Hardware Livre, extrapolou as fronteiras da computação e engenharia e hoje temos vários projetos de Hardware Livre, dentro da arquitetura e do urbanismo, na criação de máquinas entre outros. Alguns projetos interessantes serão citados abaixo com brevidade:

\subsubsection{Máquinas livres}
O site Open Source Ecology \cite{Open Source Ecology} possui um projeto chamado Machines: Global Village Construction Set \cite{Machines: Global Village Construction Set}, que consiste em uma coleção de projetos open source criados com o intuito de liberar o conhecimento sobre a fabricação máquinas industriais  para construir uma pequena e sustentável civilização com confortos modernos.

\subsubsection{Arquitetura livre}
A arquitetura livre pode ser encontrada nos sites Wikihouse \cite{Wikihouse} e no site Arc e Tec \cite{Arc e Tec}, nestes sites são compartilhados projetos de arquitetura,urbanismo e design "open source".

\subsubsection{Arduíno}
O Arduíno \cite{Arduíno} é uma ferramenta utilizada para fazer com que computadores possam "sentir" e controlar mais do mundo físico além dos computadores pessoais. Ele é uma plataforma física \textit{open-source} de computação baseada numa simples placa de microcontrolador, e um ambiente de desenvolvimento para escrever códigos para a placa.

\subsection{Sistemas embarcados}

\section{Trabalhos correlatos}
Abaixo aponto alguns trabalhos que tangem, em maior ou menor grau, algumas das premissas tomadas como objetivos deste projeto.

\subsection{Sensorium}

Sensorium é um projeto de arte, tecnologia e inovação criado pelo grupo de pesquisa Ecoarte – IHAC/Universidade Federal da Bahia - UFBA que consiste na criação de um dispositivo móvel com o uso de sensores para fazer a interação com o ambiente. O projeto Sensorium foi dividido em 4 fases, sendo elas a criação do dispositivo, uma performance com ação da comunidade, a visualização dos dados coletados e a exposição do projeto artístico e seus processos.

A criação do dispositivo envolveu prioritariamente soluções livres, tanto de \textit{hardware} quanto de \textit{software}, a facilidade operacional do dispositivo, a geração e a leitura dos dados obtidos por pessoas não ligadas a área técnica/tecnológica.

Para tanto, o prototipo foi construído com o microcontrolador arduino, utilizando os sensores de temperatura do ar e da água, umidade relativa do ar, salinidade, níveis de intensidades sonoras, níveis de CO2, GPS e Ultra violeta.

\subsection{The Copenhagen Wheel}
Este projeto foi criado pelo por um grupo no Instituto de tecnologia de Massachucets (\textit{Massachucets Institute of Thecnologic}) ou MIT no ano de 2005 tem como objetivo transformar biciletas comuns em \cite{e-bikes}.

O projeto usa o \textit{smartphone} para controlar o dispositivo que está acoplado à bicicleta.

\textit{The Copenhagen Wheel} mapeia os níveis de poluição, congestionamento do tráfego e as condicões da estrada em tempo real. 

Enquanto a bicicleta é usada, os sensores da \textit{the Copenhagen Wheel} capturaram informações sobre suas preferências pessoais de ciclismo, como quanto esforço é posto nisso, quantas calorias foram queimadas e etc assim como as informações sobre o ambiente, incluindo quantidade de monóxido de carbono, óxido de nitrogênio (NOx), níveis de ruído, temperatura ambiental e umidade relativa.

\section{Metodologia}
Primeiramente foram analisados os resultados de uma pesquisa sobre o uso da bicicleta  e fatores relativos, realizada no ano de 2012 entre pessoas de diferentes perfis e majoritariamente participantes de grupos de usuários de bicicletas em Salvador.

Analisadas essas informações, foram definidos diversos aspectos que poderiam ser aferidos através de um dispositivo eletrônico a ser anexado a uma bicicleta afim de medir alguns destes fatores.

Serão analisados fatores ambientais como:

\begin{itemize}

  \item Temperatura do ar.
  \item Umidade do ar.
  \item Luminosidade.
  \item Nível de ruído.\ldots

\end{itemize}

Assim como fatores urbanos:

\begin{itemize}

  \item Nível de trepidação da pista.
  \item Distância em que outros veículos motorizados passam pelo usuário de bicicleta.
  \item Velocidade em que outros veículos motorizados passam pelo usuário. \ldots

\end{itemize}

Após a etapa da criação do dispositivo, nós começaremos a fase de testes com o mesmo, afim de capturar e remover possíveis falhas encontradas.

Quando o dispositivo estiver maduro, iremos coletar os primeiros dados através de parcerias com grupos de ciclistas.

Por fim, analisaremos esses dados e os divulgaremos a toda a comunidade interessada.

\section{Soluções adotadas}

O desenvolvimento do dispositivo se dará através da plataforma  de \textit{hardware} livre, Arduíno, juntando se a isso os sensores necessários para a checagem das informações ambientais e urbanos. Apartir desse conjunto, os dados serão capturados, sem nenhum tipo de tratamento, e serão enviados, através de uma conexão serial\cite{Conexão Serial} para outro dispositivo, com mais capacidade de processamento, mas ainda sim portável, o RaspBerry Pi\cite{Raspberry Pi}, onde os dados serão então tratados e armazenados para posterior utilização, 

Os dados capturados serão armazenados em um contexto espacial (geolocalizados) e temporal.

\subsection{Microcontrolador}
O desenvolvimento do dispositivo será feito com o Arduíno Mega 1650 por este modelo possuir melhor custo/benefício em comparação aos outros modelos da plataforma, ele possui uma memória maior, possibilitando assim, maior flexibilidade na captura de maiores volumes de dados sem comprometer o desempenho do dispositivo, além do mesmo contar com um número maior de portas digitais e analógicas possibilitando assim, que o dispositivo seja flexível para receber novos sensores e capturar dados diferentes de acordo com a necessidade local.

\subsection{Raspberry Pi}
O Raspberry Pi, é "um computador em uma placa" do tamanho aproximado de um cartão de crédito, ele possui um processador single core com barramento de 700 MHZ, uma memória ram de 512MB, além disso, nele está acoplado um cartão SD (Secure Digital) com capacidade de 8GB e classe 10, onde estarão armazenados o sistema operacional que controla o Raspberry Pi, os scripts da aplicação responsáveis pela comunicação, recebimento, tratamento e armazenamento dos dados vindos do Arduíno.

O Raspberry Pi também possuem pinos digitais que poderiam ser utilizados para 

\subsection{Alimentação elétrica}
A alimentação deste conjunto formado pelo Arduínos com seus sensores, e o Raspberry Pi, é feita através de uma bateria com capacidade nominal de 30000 MAH, e capacidade real de 16000 MAH. Esta bateria é alimentada por uma fotocélula, enquanto a bateria estiver no sol, ela é automaticamente recarregada, com isso o projeto ganha maior autonomia, e ainda conseguimos usar uma tecnologia verde\cite{tecnologia verde} não poluente, e que se utiliza de uma fonte renovável.

\subsection{Sensor de iluminação}
Para fazer a medição da iluminação usarei um resistor dependente de luz (\textit{Light Dependent Resistor}) ou LDR. O LDR é um resistor analógico cuja resistência aplicada ao circuito varia de acordo com o nível de iluminação captado, podendo assim capturar a variação da resistência e criar com isso uma tabela de equiparação com a luz solar em diversos momentos do dia afim de tirar uma média desses valores. 

\subsection{Sensor de temperatura e umidade}
Para a medição da temperatura e da umidade do ar, irei utilizar um módulo chamado DHT11, que faz a medição de temperatura e umidade do ar de forma digital.

\subsection{Sensor de som}
O nível de ruído será capturado através de um módulo de som, que consiste em um pequeno microfone com um potênciometro para regular a sensibilidade do mesmo.

\subsection{Acelerômetro}
O nível de trepidação será capturado através de um módulo digital de um acelerômetro de 3 eixos modelo MMA7361, fixando o eito y como nível horizontal, ele capturará as variações neste eixo enquanto o dispositivo está ligado, podendo assim perceber trechos de maior variação indicando possivelmente um terreno mais acidentado e por conseguinte menos própício ao uso da bicicleta.

\subsection{Geolocalizador}
A geolocalização será feita através de um \textit{shield} GPS, \textit{global positioning system}, fabricado pela empresa ITEAD Studio. Este \textit{shield} conta também com um módulo de cartões micro-SD que será usado para armazenar os dados obtidos.

\subsection{Relógio de tempo real}
A variação temporal será medida de acordo com o módulo de relógio de tempo real(\textit{real time clock}) modelo TinyRTC v1.1.

\subsection{Sonar ultra-sônico}
A distância entre um veículo motorizado e o ciclista será medida por um sensor ultra-sônico modelo HC-SR04. Este sensor emite ondas ultra-sônicas e as recebe de volta depois que as ondas são rebatidas por algum objeto, ele então calcula a distância do objeto com base na diferença do tempo entre emissão e recebimento.

\section{Artefatos de software}
Como parte da solução desenvolvida, foi necessário também, o desenvolvimento de alguns artefatos de \textit{software} que serão apresentados nas sub-sessões abaixo.

\subsection{Implementação no Arduíno}
A linguagem de programação adotada, oficialmente, na plataforma Arduíno, é a \textit{Arduino programming language}\cite{arduinopl} que nada mais é do que um conjunto mais restrito da linguagem C++. Existe a possibilidade de usar outras linguagens de programação para programar o Arduíno, mas essas adaptações são geralmente feitas e mantidas pela comunidade. 

O projeto Arduíno mantém também uma \textit{Integrated Development Environment} ou IDE, onde é possível seguir todo o fluxo de desenvolvimento para a plataforma, desde a criação de bibliotecas, codificação da aplicação, compilação da aplicação e o processo de embarcar o binário para a placa além de possuir versão para os sistemas  operacionais Windows GNU/Linux e Mac OS X.

A minha escolha da linguagem se deu pela \textit{Arduino programming language}, por ela ser uma linguagem de médio nível, possuir boa documentação e por já possuir domínio com a mesma e a IDE foi escolhida foi a do próprio Arduíno por possuir todas as ferramentas necessárias para o desenvolvimento na plataforma,

\subsection{Scripts do Arduíno}


\subsection{Armazenamento de dados}
Para o armazenamento de dados foi feito um estudo e um teste de desempenho entre duas tecnologias com bastante uso para armazenamento de dados, a \textit{eXtensible Markup Language} ou XML e o \textit{JavaScript Object Notation} ou JSON. As duas tecnologias armazenam os dados em um arquivo de texto plano, mas possuem diferenças que trazem características únicas a cada um deles.

\section{Testes e resultados}

\label{}

%% The Appendices part is started with the command \appendix;
%% appendix sections are then done as normal sections
%% \appendix

%% \section{}
%% \label{}

%% References
%%
%% Following citation commands can be used in the body text:
%% Usage of \cite is as follows:
%%   \cite{key}          ==>>  [#]
%%   \cite[chap. 2]{key} ==>>  [#, chap. 2]
%%   \citet{key}         ==>>  Author [#]

%% References with bibTeX database:

\bibliographystyle{model1-num-names}
\bibliography{<your-bib-database>}

%% Authors are advised to submit their bibtex database files. They are
%% requested to list a bibtex style file in the manuscript if they do
%% not want to use model1-num-names.bst.

%% References without bibTeX database:

 \begin{thebibliography}{00}

%% \bibitem must have the following form:
	\bibitem{Arduíno}
	Arduíno Introduction, disponível em \url{http://www.arduino.cc/en/Guide/Introduction}. Último acesso em 26/05/2014.
	\bibitem{Wikihouse}
	Wikihouse, disponível em \url{http://www.wikihouse.cc/}. Último acesso em 26/05/2014.
   \bibitem{Open Source Ecology}
   Open Source Ecology, disponível em \url{http://opensourceecology.org/}. Último acesso em 26/05/2014.
   \bibitem{Machines: Global Village Construction Set}
   Machines: Global Village Construction Set, disponível em \url{http://opensourceecology.org/gvcs/}. Último acesso em 26/05/2014.
   \bibitem{Arc e Tec}
   Arc e Tec, disponível em \url{http://www.arq-e-tec.com/2011/09/wikihouse-projetos-de-arquitetura-open-source/}. Último acesso em 26/05/2014.
%%

% \bibitem{}

 \end{thebibliography}


\end{document}

%%
%% End of file `elsarticle-template-1-num.tex'.
